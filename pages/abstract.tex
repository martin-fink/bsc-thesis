\chapter{\abstractname}\label{ch:abstract}

With new CPU architectures such as ARM or RISC-V gaining popularity, achieving application support is a problem hindering their adaption.
Existing applications need to be adapted, compiled, and distributed for the new architecture, which is not always possible.
Static binary translation is a concept that can help port binaries to new architectures without introducing significant runtime overhead.
These translators act similar to a compiler in the sense that they process and translate binaries ahead of time.
In this thesis, we try to improve and extend the capabilities of the existing binary lifter MCTOLL\@.
It gradually recovers abstraction levels from binaries and produces LLVM bitcode, which can then be compiled to all architectures supported by LLVM\@.

To test our implementation, we compare native programs to their translated counterparts by measuring runtime performance and code size of the phoenix-2.0 benchmark.

The results show that static binary translators are able to generate very efficient code with the help of optimizers such as LLVM\@.
However, they do have some limitations and cannot process binaries with certain properties, which limits their use cases.

\chapter{Zusammenfassung}\label{ch:zusammenfassung}

Mit dem zunehmenden Aufschwung von neuen Prozessorarchitekturen wie ARM oder RISC-V stellt die mangelnde Kompatibilität bestehender Anwendungen mit diesen Architekturen ein Problem dar.
Programme müssen an die neue Architektur angepasst, neu kompiliert und an die Benutzer verteilt werden, was nicht immer möglich ist.
Statische Übersetzung von Binärdateien ist ein Ansatz, Programme auf neue Architekturen zu portieren, ohne erheblich an Laufzeit einzubüßen.
Diese Übersetzer verhalten sich ähnlich wie klassische Compiler indem sie Programme verarbeiten, bevor sie ausgeführt werden.
In dieser Arbeit versuchen wir die Fähigkeiten des bestehenden Übsersetzers MCTOLL zu verbessern und erweitern.
MCTOLL versucht schrittweise Abstraktionsebenen des ursprünglichen Programmcodes zurückzugewinnen und erzeugt LLVM Bitcode, welcher von LLVM dann für alle unterstützten Zielplattformen kompiliert werden kann.

Um die Ergebnisse unserer Arbeit zu testen, vergleichen wir native Programme mit den übersetzten und portierten Gegenstücken.
Wir messen die Laufzeit und Programmgröße des phoenix-2.0 Benchmarks.
Die Ergebnisse zeigen, dass statische Binärübersetzer mit der Hilfe von Codeoptimierern wie LLVM in der Lage sind, sehr effizienten Code zu erzeugen.
Es gibt jedoch Einschränkungen in der Anwendbarkeit, da diese Typen von Binärübersetzern nicht jede Art von Programm verarbeiten können.
